% Options for packages loaded elsewhere
\PassOptionsToPackage{unicode}{hyperref}
\PassOptionsToPackage{hyphens}{url}
%
\documentclass[
]{article}
\usepackage{lmodern}
\usepackage{amssymb,amsmath}
\usepackage{ifxetex,ifluatex}
\ifnum 0\ifxetex 1\fi\ifluatex 1\fi=0 % if pdftex
  \usepackage[T1]{fontenc}
  \usepackage[utf8]{inputenc}
  \usepackage{textcomp} % provide euro and other symbols
\else % if luatex or xetex
  \usepackage{unicode-math}
  \defaultfontfeatures{Scale=MatchLowercase}
  \defaultfontfeatures[\rmfamily]{Ligatures=TeX,Scale=1}
\fi
% Use upquote if available, for straight quotes in verbatim environments
\IfFileExists{upquote.sty}{\usepackage{upquote}}{}
\IfFileExists{microtype.sty}{% use microtype if available
  \usepackage[]{microtype}
  \UseMicrotypeSet[protrusion]{basicmath} % disable protrusion for tt fonts
}{}
\makeatletter
\@ifundefined{KOMAClassName}{% if non-KOMA class
  \IfFileExists{parskip.sty}{%
    \usepackage{parskip}
  }{% else
    \setlength{\parindent}{0pt}
    \setlength{\parskip}{6pt plus 2pt minus 1pt}}
}{% if KOMA class
  \KOMAoptions{parskip=half}}
\makeatother
\usepackage{xcolor}
\IfFileExists{xurl.sty}{\usepackage{xurl}}{} % add URL line breaks if available
\IfFileExists{bookmark.sty}{\usepackage{bookmark}}{\usepackage{hyperref}}
\hypersetup{
  pdftitle={Proyecto semestral: Hito 1},
  pdfauthor={Nicolás Herrera, Yesenia Marulanda, Franco Migliorelli, Samuel Sánchez, Sebastián Urbina},
  hidelinks,
  pdfcreator={LaTeX via pandoc}}
\urlstyle{same} % disable monospaced font for URLs
\usepackage[margin=1in]{geometry}
\usepackage{color}
\usepackage{fancyvrb}
\newcommand{\VerbBar}{|}
\newcommand{\VERB}{\Verb[commandchars=\\\{\}]}
\DefineVerbatimEnvironment{Highlighting}{Verbatim}{commandchars=\\\{\}}
% Add ',fontsize=\small' for more characters per line
\usepackage{framed}
\definecolor{shadecolor}{RGB}{248,248,248}
\newenvironment{Shaded}{\begin{snugshade}}{\end{snugshade}}
\newcommand{\AlertTok}[1]{\textcolor[rgb]{0.94,0.16,0.16}{#1}}
\newcommand{\AnnotationTok}[1]{\textcolor[rgb]{0.56,0.35,0.01}{\textbf{\textit{#1}}}}
\newcommand{\AttributeTok}[1]{\textcolor[rgb]{0.77,0.63,0.00}{#1}}
\newcommand{\BaseNTok}[1]{\textcolor[rgb]{0.00,0.00,0.81}{#1}}
\newcommand{\BuiltInTok}[1]{#1}
\newcommand{\CharTok}[1]{\textcolor[rgb]{0.31,0.60,0.02}{#1}}
\newcommand{\CommentTok}[1]{\textcolor[rgb]{0.56,0.35,0.01}{\textit{#1}}}
\newcommand{\CommentVarTok}[1]{\textcolor[rgb]{0.56,0.35,0.01}{\textbf{\textit{#1}}}}
\newcommand{\ConstantTok}[1]{\textcolor[rgb]{0.00,0.00,0.00}{#1}}
\newcommand{\ControlFlowTok}[1]{\textcolor[rgb]{0.13,0.29,0.53}{\textbf{#1}}}
\newcommand{\DataTypeTok}[1]{\textcolor[rgb]{0.13,0.29,0.53}{#1}}
\newcommand{\DecValTok}[1]{\textcolor[rgb]{0.00,0.00,0.81}{#1}}
\newcommand{\DocumentationTok}[1]{\textcolor[rgb]{0.56,0.35,0.01}{\textbf{\textit{#1}}}}
\newcommand{\ErrorTok}[1]{\textcolor[rgb]{0.64,0.00,0.00}{\textbf{#1}}}
\newcommand{\ExtensionTok}[1]{#1}
\newcommand{\FloatTok}[1]{\textcolor[rgb]{0.00,0.00,0.81}{#1}}
\newcommand{\FunctionTok}[1]{\textcolor[rgb]{0.00,0.00,0.00}{#1}}
\newcommand{\ImportTok}[1]{#1}
\newcommand{\InformationTok}[1]{\textcolor[rgb]{0.56,0.35,0.01}{\textbf{\textit{#1}}}}
\newcommand{\KeywordTok}[1]{\textcolor[rgb]{0.13,0.29,0.53}{\textbf{#1}}}
\newcommand{\NormalTok}[1]{#1}
\newcommand{\OperatorTok}[1]{\textcolor[rgb]{0.81,0.36,0.00}{\textbf{#1}}}
\newcommand{\OtherTok}[1]{\textcolor[rgb]{0.56,0.35,0.01}{#1}}
\newcommand{\PreprocessorTok}[1]{\textcolor[rgb]{0.56,0.35,0.01}{\textit{#1}}}
\newcommand{\RegionMarkerTok}[1]{#1}
\newcommand{\SpecialCharTok}[1]{\textcolor[rgb]{0.00,0.00,0.00}{#1}}
\newcommand{\SpecialStringTok}[1]{\textcolor[rgb]{0.31,0.60,0.02}{#1}}
\newcommand{\StringTok}[1]{\textcolor[rgb]{0.31,0.60,0.02}{#1}}
\newcommand{\VariableTok}[1]{\textcolor[rgb]{0.00,0.00,0.00}{#1}}
\newcommand{\VerbatimStringTok}[1]{\textcolor[rgb]{0.31,0.60,0.02}{#1}}
\newcommand{\WarningTok}[1]{\textcolor[rgb]{0.56,0.35,0.01}{\textbf{\textit{#1}}}}
\usepackage{graphicx,grffile}
\makeatletter
\def\maxwidth{\ifdim\Gin@nat@width>\linewidth\linewidth\else\Gin@nat@width\fi}
\def\maxheight{\ifdim\Gin@nat@height>\textheight\textheight\else\Gin@nat@height\fi}
\makeatother
% Scale images if necessary, so that they will not overflow the page
% margins by default, and it is still possible to overwrite the defaults
% using explicit options in \includegraphics[width, height, ...]{}
\setkeys{Gin}{width=\maxwidth,height=\maxheight,keepaspectratio}
% Set default figure placement to htbp
\makeatletter
\def\fps@figure{htbp}
\makeatother
\setlength{\emergencystretch}{3em} % prevent overfull lines
\providecommand{\tightlist}{%
  \setlength{\itemsep}{0pt}\setlength{\parskip}{0pt}}
\setcounter{secnumdepth}{-\maxdimen} % remove section numbering

\title{Proyecto semestral: Hito 1}
\author{Nicolás Herrera, Yesenia Marulanda, Franco Migliorelli, Samuel Sánchez,
Sebastián Urbina}
\date{Octubre 2020}

\begin{document}
\maketitle

{
\setcounter{tocdepth}{2}
\tableofcontents
}
\hypertarget{motivacion}{%
\section{Motivacion:}\label{motivacion}}

Twitter es una de las redes sociales más utilizadas para comentar,
compartir o debatir temas de actualidad y tendencias. El primer
trimestre de 2020 tuvo un aumento de un 23\% en comentarios diarios con
respecto al mismo periodo de tiempo del año 2019 {[}1{]}, siendo una de
sus principales causas la pandemia del COVID-19.

Es por este contexto que analizar datos de twitter es interesante,
además se puede obtener información en tiempo real de sucesos que están
ocurriendo, se pueden ocupar métodos de la minería de datos sobre esta
información y es fácil manipular grandes volúmenes de información. Por
estas razones, nace la motivación por explorar tweets relacionados con
coronavirus en un intervalo de tiempo acotado, desde la perspectiva de
los sentimientos y relacionandolos con el contexto pais desde donde se
emiten; buscando establecer si este ùltimo influencia la percepcion de
las personas acerca de la pandemia.

Dicho lo anterior, los puntos claves a analizar en el desarrollo del
proyecto son:

\begin{itemize}
\tightlist
\item
  Identificar como se relaciona el sentimiento identificado con el
  contexto pais (segun mayor o menor presencia del sentimiento).
\item
  Categorizar paises segun mayor o menor presencia de sentimiento y
  relacionarlos con algun indice de felicidad publicado en el ultimo
  año.
\item
  Identificar palabras que son clave a la hora de categorizar el
  sentimiento.
\item
  Establecer algoritmos para predecir sentimientos de forma
  sistematizada.
\item
  Entrenar modelos de clasificaciòn en base a tweets usando un Dataset
  de entrenamiento y un dataset de evaluacion.
\end{itemize}

\hypertarget{descripciuxf3n-de-base-de-datos}{%
\section{Descripción de base de
datos}\label{descripciuxf3n-de-base-de-datos}}

La base de datos, extraída desde la plataforma \emph{Kaggle}{[}2{]},
esta compuesta en principio por 44955 tweets relacionados con el tema
COVID 19 y que fueron publicados del 2 de marzo al 14 de abril de 2020.
Ademas de encontrarse el texto publicado se encuentran en la base de
datos los atributos de fecha exacta de publicacion, ubicacion desde la
cual se realizla publicacion, un identificador para el usuario y la
asignacion de sentimiento para cada tweet. La asignacion de sentimiento
a cada Tweet fue realizada de forma manual por el propietario de la base
de datos. Esta variable de sentimiento sería un punto de comparación
para un posible modelo de clasifiación de los sentimientos de los
tweets.

\hypertarget{exploraciuxf3n-de-datos}{%
\section{Exploración de datos}\label{exploraciuxf3n-de-datos}}

El objetivo de esta sección es describir la base de datos seleccionadas,
mostrando estadísticas de resumen, gráficos relevantes para la
descripción y un breve análisis sobre estos.

El primer paso consiste en cargar todas las librerías que se utilizarán
en este trabajo, las cuales permiten realizar gráficos y trabajar con
los datos de una forma más simple y eficiente.

\begin{Shaded}
\begin{Highlighting}[]
\KeywordTok{library}\NormalTok{(ggplot2)}
\KeywordTok{library}\NormalTok{(dplyr)}
\KeywordTok{library}\NormalTok{(tidyverse)}
\KeywordTok{library}\NormalTok{(tidytext)}
\KeywordTok{library}\NormalTok{(stopwords)}
\KeywordTok{library}\NormalTok{(wordcloud)}
\KeywordTok{library}\NormalTok{(wordcloud2)}
\KeywordTok{library}\NormalTok{(stringr)}
\end{Highlighting}
\end{Shaded}

A continuación se procece a cargar la base de datos para poder realizar
el análisis respectivo.

\begin{Shaded}
\begin{Highlighting}[]
\NormalTok{train  <-}\StringTok{ }\KeywordTok{read.csv}\NormalTok{(}\StringTok{"data/Corona_NLP_train.csv"}\NormalTok{, }\DataTypeTok{encoding=}\StringTok{"Latin-1"}\NormalTok{)}
\NormalTok{test <-}\StringTok{ }\KeywordTok{read.csv}\NormalTok{(}\StringTok{"data/Corona_NLP_test.csv"}\NormalTok{, }\DataTypeTok{encoding=}\StringTok{"Latin-1"}\NormalTok{)}
\NormalTok{df <-}\StringTok{ }\KeywordTok{rbind}\NormalTok{(train,test)}
\end{Highlighting}
\end{Shaded}

Tal y como se mencionó anteriormente, esta base de datos contiene 6
columnas y 44955 filas de datos. Para poder tener una referencia de cómo
son estos campos, a continuación se puede observar una vista previa de
las primeras filas del set de datos.

\begin{Shaded}
\begin{Highlighting}[]
\KeywordTok{head}\NormalTok{(df)}
\end{Highlighting}
\end{Shaded}

\begin{verbatim}
##   UserName ScreenName                  Location    TweetAt
## 1     3799      48751                    London 16-03-2020
## 2     3800      48752                        UK 16-03-2020
## 3     3801      48753                 Vagabonds 16-03-2020
## 4     3802      48754                           16-03-2020
## 5     3803      48755                           16-03-2020
## 6     3804      48756 ÃœT: 36.319708,-82.363649 16-03-2020
##                                                                                                                                                                                                                                                                                                        OriginalTweet
## 1                                                                                                                                                                                                    @MeNyrbie @Phil_Gahan @Chrisitv https://t.co/iFz9FAn2Pa and https://t.co/xX6ghGFzCC and https://t.co/I2NlzdxNo8
## 2                                                                      advice Talk to your neighbours family to exchange phone numbers create contact list with phone numbers of neighbours schools employer chemist GP set up online shopping accounts if poss adequate supplies of regular meds but not over order
## 3                                                                                                                                                                                Coronavirus Australia: Woolworths to give elderly, disabled dedicated shopping hours amid COVID-19 outbreak https://t.co/bInCA9Vp8P
## 4      My food stock is not the only one which is empty...\n\nPLEASE, don't panic, THERE WILL BE ENOUGH FOOD FOR EVERYONE if you do not take more than you need. \nStay calm, stay safe.\n\n#COVID19france #COVID_19 #COVID19 #coronavirus #confinement #Confinementotal #ConfinementGeneral https://t.co/zrlG0Z520j
## 5 Me, ready to go at supermarket during the #COVID19 outbreak.\n\nNot because I'm paranoid, but because my food stock is litteraly empty. The #coronavirus is a serious thing, but please, don't panic. It causes shortage...\n\n#CoronavirusFrance #restezchezvous #StayAtHome #confinement https://t.co/usmuaLq72n
## 6                                                         As news of the regionÂ’s first confirmed COVID-19 case came out of Sullivan County last week, people flocked to area stores to purchase cleaning supplies, hand sanitizer, food, toilet paper and other goods, @Tim_Dodson reports https://t.co/cfXch7a2lU
##            Sentiment
## 1            Neutral
## 2           Positive
## 3           Positive
## 4           Positive
## 5 Extremely Negative
## 6           Positive
\end{verbatim}

Para poder trabajar con los datos es necesario convertir algunos
formatos de las variables, en particular en este caso, se procede a
transformar las variables ``TweetAt'' a un formato de fecha, dado que
corresponde al momento en donde se realizó el tweet, y la variable
``OriginalTweet'' que corresponde al contenido del tweet realizado.

\begin{Shaded}
\begin{Highlighting}[]
\NormalTok{df}\OperatorTok{$}\NormalTok{TweetAt <-}\StringTok{ }\KeywordTok{as.Date}\NormalTok{(df}\OperatorTok{$}\NormalTok{TweetAt, }\DataTypeTok{format=}\StringTok{"%d-%m-%y"}\NormalTok{)}
\NormalTok{df}\OperatorTok{$}\NormalTok{OriginalTweet <-}\StringTok{ }\KeywordTok{as.character}\NormalTok{(df}\OperatorTok{$}\NormalTok{OriginalTweet)}
\end{Highlighting}
\end{Shaded}

Los formatos de cada variable del set de datos son mostrados a
continuación:

\begin{Shaded}
\begin{Highlighting}[]
\KeywordTok{str}\NormalTok{(df)}
\end{Highlighting}
\end{Shaded}

\begin{verbatim}
## 'data.frame':    44955 obs. of  6 variables:
##  $ UserName     : int  3799 3800 3801 3802 3803 3804 3805 3806 3807 3808 ...
##  $ ScreenName   : int  48751 48752 48753 48754 48755 48756 48757 48758 48759 48760 ...
##  $ Location     : chr  "London" "UK" "Vagabonds" "" ...
##  $ TweetAt      : Date, format: "2020-03-16" "2020-03-16" ...
##  $ OriginalTweet: chr  "@MeNyrbie @Phil_Gahan @Chrisitv https://t.co/iFz9FAn2Pa and https://t.co/xX6ghGFzCC and https://t.co/I2NlzdxNo8" "advice Talk to your neighbours family to exchange phone numbers create contact list with phone numbers of neigh"| __truncated__ "Coronavirus Australia: Woolworths to give elderly, disabled dedicated shopping hours amid COVID-19 outbreak htt"| __truncated__ "My food stock is not the only one which is empty...\n\nPLEASE, don't panic, THERE WILL BE ENOUGH FOOD FOR EVERY"| __truncated__ ...
##  $ Sentiment    : chr  "Neutral" "Positive" "Positive" "Positive" ...
\end{verbatim}

Además, es importante mencionar con qué periodos se estará trabajando,
por lo que el resultado de la siguiente línea de código arroja la fecha
mínima y máxima del set de datos, siendo el 2 de marzo de 2020 y el 14
de abril de 2020 respectivamente.

\begin{Shaded}
\begin{Highlighting}[]
\KeywordTok{summary}\NormalTok{(df}\OperatorTok{$}\NormalTok{TweetAt)}
\end{Highlighting}
\end{Shaded}

\begin{verbatim}
##         Min.      1st Qu.       Median         Mean      3rd Qu.         Max. 
## "2020-03-02" "2020-03-19" "2020-03-23" "2020-03-26" "2020-04-06" "2020-04-14"
\end{verbatim}

Para ver cómo están distribuidos los tweets en el tiempo, a continuación
se realiza y muestra un histograma con las distribuciones de las fechas
de los tweets realizados y registrados en esta base de datos.

\begin{Shaded}
\begin{Highlighting}[]
\KeywordTok{ggplot}\NormalTok{(}\DataTypeTok{data=}\NormalTok{df, }\KeywordTok{aes}\NormalTok{(}\DataTypeTok{x=}\NormalTok{TweetAt)) }\OperatorTok{+}\StringTok{ }\KeywordTok{geom_histogram}\NormalTok{(}\DataTypeTok{position=}\StringTok{"identity"}\NormalTok{, }\DataTypeTok{bins=}\DecValTok{30}\NormalTok{) }\OperatorTok{+}
\StringTok{  }\KeywordTok{labs}\NormalTok{(}\DataTypeTok{title =} \StringTok{"Distribución de las fechas de tweets"}\NormalTok{, }\DataTypeTok{x =} \StringTok{"fecha de publicación",}
\StringTok{       y = "}\NormalTok{número de tweets}\StringTok{") + theme_bw()}
\end{Highlighting}
\end{Shaded}

\includegraphics{hito_1_files/figure-latex/unnamed-chunk-7-1.pdf}

Un comportamiento particular de los datos es que a finales de marzo la
cantidad de tweets registrados disminiye considerablemente llegando a
ser nulo, mientras que la mayor cantidad de tweets se encuentra
concentrada durante la tercera semana de marzo y la segunda semana de
abril.

Un campo importante que posee esta base de datos es la columna
``Sentiment'', la cual representa el sentimiento asociado al tweet
registrado. A continuación se presenta la cantidad de tweets segun tipo
de sentimiento (``Extremely Negative'', ``Extremely Postive'',
``Negative'', ``Neutral'', ``Positive'') durante el periodo registrado
en el set de datos.

\begin{Shaded}
\begin{Highlighting}[]
\NormalTok{tweets_mes_dia <-}\StringTok{ }\NormalTok{df }\OperatorTok\StringTok{ }\KeywordTok{mutate}\NormalTok{(}\DataTypeTok{mes_dia =} \KeywordTok{format}\NormalTok{(TweetAt, }\StringTok{"%m-%d"}\NormalTok{))}
\NormalTok{tweets_mes_dia }\OperatorTok\StringTok{ }\KeywordTok{group_by}\NormalTok{(Sentiment, mes_dia) }\OperatorTok\StringTok{ }\KeywordTok{summarise}\NormalTok{(}\DataTypeTok{n =} \KeywordTok{n}\NormalTok{()) }\OperatorTok
\StringTok{  }\KeywordTok{ggplot}\NormalTok{(}\KeywordTok{aes}\NormalTok{(}\DataTypeTok{x =}\NormalTok{ mes_dia, }\DataTypeTok{y =}\NormalTok{ n, }\DataTypeTok{color =}\NormalTok{ Sentiment)) }\OperatorTok{+}
\StringTok{  }\KeywordTok{geom_line}\NormalTok{(}\KeywordTok{aes}\NormalTok{(}\DataTypeTok{group =}\NormalTok{ Sentiment)) }\OperatorTok{+}
\StringTok{  }\KeywordTok{labs}\NormalTok{(}\DataTypeTok{title =} \StringTok{"Número de tweets publicados"}\NormalTok{, }\DataTypeTok{x =} \StringTok{"fecha de publicación",}
\StringTok{       y = "}\NormalTok{número de tweets}\StringTok{") +}
\StringTok{  theme_bw() +}
\StringTok{  theme(axis.text.x = element_text(angle = 90, size = 6),}
\StringTok{        legend.position = "}\NormalTok{bottom}\StringTok{")}
\end{Highlighting}
\end{Shaded}

\begin{verbatim}
## `summarise()` regrouping output by 'Sentiment' (override with `.groups` argument)
\end{verbatim}

\includegraphics{hito_1_files/figure-latex/unnamed-chunk-8-1.pdf}

Se puede observar en este último gráfico que en general aquellos
sentimientos que prevalecen son los de ``Positive'' y ``Negative''. Por
otro lado, todos los tipos de sentimientos registrados tienen un
comportamiento similar a lo largo del tiempo. Mientras que el
sentimiento menos registrados en el tiempo corresponde al de ``Extremely
Negative'', lo cual es interesante dado el contexto de la base de datos,
en donde se esperaría que hubiese una mayor cantidad de tweets asociado
a un sentimiento negativo o extramademente negativo.

En el siguiente gráfico se puede observar la cantidad de tweets por cada
tipo de sentimiento durante todo el periodo de observación.

\begin{Shaded}
\begin{Highlighting}[]
\CommentTok{# library(tidyverse)}
\CommentTok{#tweets_sentiment <- df %>% group_by(Sentiment) %>% summarise(n = n())}
\NormalTok{df }\OperatorTok\StringTok{ }\KeywordTok{ggplot}\NormalTok{(}\KeywordTok{aes}\NormalTok{(}\DataTypeTok{x =}\NormalTok{ Sentiment)) }\OperatorTok{+}\StringTok{ }\KeywordTok{geom_bar}\NormalTok{(}\DataTypeTok{stat=}\StringTok{"count"}\NormalTok{) }\OperatorTok{+}\StringTok{ }\KeywordTok{coord_flip}\NormalTok{() }\OperatorTok{+}\StringTok{ }\KeywordTok{theme_bw}\NormalTok{()}
\end{Highlighting}
\end{Shaded}

\includegraphics{hito_1_files/figure-latex/unnamed-chunk-9-1.pdf}

Se observa que tal y como se mencionó anteriormente, los sentimientos
que destacan son ``Positive'' y ``Negative'', alcanzando un total
aproximado de 13000 y 11000 tweets respectivamente. El sentimiento con
una menor cantidad de registros es ``Extremely Negative'', con un
aproximado de 6000 tweets.

Si se comparación la proporción de estos tweets en comparación al total
del periodo de observación, se tiene que el porcentaje asociado a cada
sentimiento son los siguientes:

\begin{Shaded}
\begin{Highlighting}[]
\NormalTok{df }\OperatorTok\StringTok{ }\KeywordTok{group_by}\NormalTok{(Sentiment) }\OperatorTok\StringTok{ }\KeywordTok{summarise}\NormalTok{(}\DataTypeTok{Proporcion =} \KeywordTok{n}\NormalTok{()}\OperatorTok{/}\KeywordTok{nrow}\NormalTok{(df)) }\OperatorTok\StringTok{ }\KeywordTok{arrange}\NormalTok{(}\OperatorTok{-}\NormalTok{Proporcion)}
\end{Highlighting}
\end{Shaded}

\begin{verbatim}
## `summarise()` ungrouping output (override with `.groups` argument)
\end{verbatim}

\begin{verbatim}
## # A tibble: 5 x 2
##   Sentiment          Proporcion
##   <chr>                   <dbl>
## 1 Positive                0.275
## 2 Negative                0.244
## 3 Neutral                 0.185
## 4 Extremely Positive      0.161
## 5 Extremely Negative      0.135
\end{verbatim}

Es decir, el sentimiento ``Positive'' representa al 27,5\% de los tweets
del set de datos, mientras que el 13,5\% de los tweets están
categorizados bajo el sentimiento ``Extremely Negative''.Es importante
notar que los sentimientos extremos tanto positivo como negativo se
presentan en menor proporcion.

Otro punto importante a caracterizar es la locación en donde se emiten
los tweets. En la siguiente tabla, se puede observar la cantidad total
de tweets registrados para el top 10 de localidades.

\begin{Shaded}
\begin{Highlighting}[]
\NormalTok{top_}\DecValTok{10}\NormalTok{ <-}\StringTok{ }\NormalTok{df }\OperatorTok\StringTok{ }\KeywordTok{group_by}\NormalTok{(Location) }\OperatorTok\StringTok{ }\KeywordTok{summarise}\NormalTok{(}\DataTypeTok{N=}\KeywordTok{n}\NormalTok{()}\OperatorTok{/}\KeywordTok{nrow}\NormalTok{(df)) }\OperatorTok\StringTok{ }\KeywordTok{arrange}\NormalTok{(}\OperatorTok{-}\NormalTok{N)}
\end{Highlighting}
\end{Shaded}

\begin{verbatim}
## `summarise()` ungrouping output (override with `.groups` argument)
\end{verbatim}

\begin{Shaded}
\begin{Highlighting}[]
\NormalTok{top_}\DecValTok{10}\NormalTok{ <-}\StringTok{ }\NormalTok{top_}\DecValTok{10}\NormalTok{[}\DecValTok{1}\OperatorTok{:}\DecValTok{10}\NormalTok{,]}
\NormalTok{top_}\DecValTok{10}\OperatorTok{$}\NormalTok{N <-}\StringTok{ }\KeywordTok{round}\NormalTok{(top_}\DecValTok{10}\OperatorTok{$}\NormalTok{N,}\DecValTok{4}\NormalTok{)}
\KeywordTok{ggplot}\NormalTok{(top_}\DecValTok{10}\NormalTok{,}\KeywordTok{aes}\NormalTok{(}\DataTypeTok{x=}\KeywordTok{reorder}\NormalTok{(Location,N), }\DataTypeTok{y =}\NormalTok{N )) }\OperatorTok{+}\StringTok{ }
\StringTok{  }\KeywordTok{geom_bar}\NormalTok{(}\DataTypeTok{stat=}\StringTok{"identity"}\NormalTok{) }\OperatorTok{+}\StringTok{ }
\StringTok{  }\KeywordTok{coord_flip}\NormalTok{() }\OperatorTok{+}\StringTok{ }
\StringTok{  }\KeywordTok{labs}\NormalTok{(}\DataTypeTok{x=}\StringTok{"Pais"}\NormalTok{, }\DataTypeTok{y =} \StringTok{"Proporción del total de tweets"}\NormalTok{, }\DataTypeTok{title=}\StringTok{"Top-10 locaciones con más tweets"}\NormalTok{) }\OperatorTok{+}
\StringTok{  }\KeywordTok{theme_bw}\NormalTok{() }
\end{Highlighting}
\end{Shaded}

\includegraphics{hito_1_files/figure-latex/unnamed-chunk-11-1.pdf}

En este último gráfico se puede observar que un 20\% de los tweets no
registran alguna locación, lo que se puede atribuir a que dentro de la
aplicacion de twitter no todas las personas comparten su ubicacion. Del
porcentaje restante, los lugares más comunes son de países como Estados
Unidos, Reino Unido e India. Se puede observar tambien que la
estandarizacion de las localidades no es la mejor, pues no siempre estan
relacionadas a un pais de origen.

Para poder analizar los contenidos de los tweets es necesario realizar
una limpieza y normalización de estos. A continuación se crea una
función que permite corregir algunos patrones de los textos tales como
números, puntuación y espacios en blanco.

\begin{Shaded}
\begin{Highlighting}[]
\NormalTok{limpiar_texto <-}\StringTok{ }\ControlFlowTok{function}\NormalTok{(texto)\{}
    \CommentTok{# Se convierte todo el texto a minúsculas}
\NormalTok{    nuevo_texto <-}\StringTok{ }\KeywordTok{tolower}\NormalTok{(texto)}
    \CommentTok{# Eliminación de páginas web (palabras que empiezan por "http." seguidas }
    \CommentTok{# de cualquier cosa que no sea un espacio)}
\NormalTok{    nuevo_texto <-}\StringTok{ }\KeywordTok{str_replace_all}\NormalTok{(nuevo_texto,}\StringTok{"http}\CharTok{\textbackslash{}\textbackslash{}}\StringTok{S*"}\NormalTok{, }\StringTok{""}\NormalTok{)}
    \CommentTok{# Eliminación de signos de puntuación}
\NormalTok{    nuevo_texto <-}\StringTok{ }\KeywordTok{str_replace_all}\NormalTok{(nuevo_texto,}\StringTok{"[[:punct:]]"}\NormalTok{, }\StringTok{" "}\NormalTok{)}
    \CommentTok{# Eliminación de números}
\NormalTok{    nuevo_texto <-}\StringTok{ }\KeywordTok{str_replace_all}\NormalTok{(nuevo_texto,}\StringTok{"[[:digit:]]"}\NormalTok{, }\StringTok{" "}\NormalTok{)}
    \CommentTok{# Eliminación de espacios en blanco múltiples}
\NormalTok{    nuevo_texto <-}\StringTok{ }\KeywordTok{str_replace_all}\NormalTok{(nuevo_texto,}\StringTok{"[}\CharTok{\textbackslash{}\textbackslash{}}\StringTok{s]+"}\NormalTok{, }\StringTok{" "}\NormalTok{)}
    \CommentTok{# Tokenización por palabras individuales}
\NormalTok{    nuevo_texto <-}\StringTok{ }\KeywordTok{str_split}\NormalTok{(nuevo_texto, }\StringTok{" "}\NormalTok{)[[}\DecValTok{1}\NormalTok{]]}
    \CommentTok{# Eliminación de tokens con una longitud < 2}
\NormalTok{    nuevo_texto <-}\StringTok{ }\KeywordTok{keep}\NormalTok{(}\DataTypeTok{.x =}\NormalTok{ nuevo_texto, }\DataTypeTok{.p =} \ControlFlowTok{function}\NormalTok{(x)\{}\KeywordTok{str_length}\NormalTok{(x) }\OperatorTok{>}\StringTok{ }\DecValTok{1}\NormalTok{\})}
    \KeywordTok{return}\NormalTok{(nuevo_texto)}
\NormalTok{\}}
\end{Highlighting}
\end{Shaded}

Para entender como procede esta última función, en la siguiente línea se
muestra un ejemplo junto con los resultados de la aplicación de esta, en
donde se puede observar que se extrajo cada palabra del objeto
\emph{text}.

\begin{Shaded}
\begin{Highlighting}[]
\NormalTok{text =}\StringTok{ "Hola mi nombre es https://www.google.cl. Como. no sé xd6666 ASDA"}
\KeywordTok{limpiar_texto}\NormalTok{(text)}
\end{Highlighting}
\end{Shaded}

\begin{verbatim}
## [1] "hola"   "mi"     "nombre" "es"     "como"   "no"     "sé"     "xd"    
## [9] "asda"
\end{verbatim}

En el siguiente paso, se aplica la función \emph{limpiar\_texto} al
contenido de los tweets de la base de datos, en donde cada resultado de
cada tweet es almacenado en un vector de palabras, por lo que cada tweet
tendría asociado uno de estos vectores.

\begin{Shaded}
\begin{Highlighting}[]
\NormalTok{tweets <-}\StringTok{ }\NormalTok{df }\OperatorTok\StringTok{ }\KeywordTok{mutate}\NormalTok{(}\DataTypeTok{texto_vector =} \KeywordTok{map}\NormalTok{(}\DataTypeTok{.x =}\NormalTok{ OriginalTweet, }\DataTypeTok{.f =}\NormalTok{ limpiar_texto))}
\end{Highlighting}
\end{Shaded}

\begin{Shaded}
\begin{Highlighting}[]
\NormalTok{tweets }\OperatorTok\StringTok{ }\KeywordTok{select}\NormalTok{(texto_vector) }\OperatorTok\StringTok{ }\KeywordTok{head}\NormalTok{()}
\end{Highlighting}
\end{Shaded}

\begin{verbatim}
##                                                                                                                                                                                                                                                                                      texto_vector
## 1                                                                                                                                                                                                                                                       menyrbie, phil, gahan, chrisitv, and, and
## 2              advice, talk, to, your, neighbours, family, to, exchange, phone, numbers, create, contact, list, with, phone, numbers, of, neighbours, schools, employer, chemist, gp, set, up, online, shopping, accounts, if, poss, adequate, supplies, of, regular, meds, but, not, over, order
## 3                                                                                                                                                                              coronavirus, australia, woolworths, to, give, elderly, disabled, dedicated, shopping, hours, amid, covid, outbreak
## 4 my, food, stock, is, not, the, only, one, which, is, empty, please, don, panic, there, will, be, enough, food, for, everyone, if, you, do, not, take, more, than, you, need, stay, calm, stay, safe, covid, france, covid, covid, coronavirus, confinement, confinementotal, confinementgeneral
## 5  me, ready, to, go, at, supermarket, during, the, covid, outbreak, not, because, paranoid, but, because, my, food, stock, is, litteraly, empty, the, coronavirus, is, serious, thing, but, please, don, panic, it, causes, shortage, coronavirusfrance, restezchezvous, stayathome, confinement
## 6                                      as, news, of, the, regionâ, first, confirmed, covid, case, came, out, of, sullivan, county, last, week, people, flocked, to, area, stores, to, purchase, cleaning, supplies, hand, sanitizer, food, toilet, paper, and, other, goods, tim, dodson, reports
\end{verbatim}

\begin{Shaded}
\begin{Highlighting}[]
\CommentTok{#Cada valor de la columna texto_vector es un vector con cada palabra del texto}
\NormalTok{tweets}\OperatorTok{$}\NormalTok{texto_vector[}\DecValTok{1}\NormalTok{]}
\end{Highlighting}
\end{Shaded}

\begin{verbatim}
## [[1]]
## [1] "menyrbie" "phil"     "gahan"    "chrisitv" "and"      "and"
\end{verbatim}

En donde cada valor de la columna texto\_vector es un vector con cada
palabra del texto

\begin{Shaded}
\begin{Highlighting}[]
\CommentTok{#unnest() nos permite realizar una expansión de los vectores de palabras que creamos, esto aumenta la dimension de filas considerablemente}
\NormalTok{tweets_expand <-}\StringTok{ }\NormalTok{tweets }\OperatorTok\StringTok{ }\KeywordTok{select}\NormalTok{(}\OperatorTok{-}\NormalTok{OriginalTweet) }\OperatorTok\StringTok{ }\KeywordTok{unnest}\NormalTok{()}
\end{Highlighting}
\end{Shaded}

\begin{verbatim}
## Warning: `cols` is now required when using unnest().
## Please use `cols = c(texto_vector)`
\end{verbatim}

\begin{Shaded}
\begin{Highlighting}[]
\NormalTok{tweets_expand <-}\StringTok{ }\NormalTok{tweets_expand }\OperatorTok\StringTok{ }\KeywordTok{rename}\NormalTok{(}\DataTypeTok{word =}\NormalTok{ texto_vector)}
\KeywordTok{head}\NormalTok{(tweets_expand) }
\end{Highlighting}
\end{Shaded}

\begin{verbatim}
## # A tibble: 6 x 6
##   UserName ScreenName Location TweetAt    Sentiment word    
##      <int>      <int> <chr>    <date>     <chr>     <chr>   
## 1     3799      48751 London   2020-03-16 Neutral   menyrbie
## 2     3799      48751 London   2020-03-16 Neutral   phil    
## 3     3799      48751 London   2020-03-16 Neutral   gahan   
## 4     3799      48751 London   2020-03-16 Neutral   chrisitv
## 5     3799      48751 London   2020-03-16 Neutral   and     
## 6     3799      48751 London   2020-03-16 Neutral   and
\end{verbatim}

Se utilizan \emph{stopwords} para filtrar algunas palabras propias del
ingles (lenguaje dominio de los comentarios) como artículos, pronombres,
preposiciones, adverbios e incluso algunos verbos. Palabras que no
tienen un significado por si solas, sino que modifican o acompañan a
otras.

\begin{Shaded}
\begin{Highlighting}[]
\CommentTok{# "word" %in% vector -> true or false}
\NormalTok{lista_stopwords <-}\StringTok{ }\KeywordTok{stopwords}\NormalTok{(}\StringTok{"english"}\NormalTok{)}
\NormalTok{lista_stopwords <-}\StringTok{ }\KeywordTok{c}\NormalTok{(lista_stopwords, }\StringTok{"amp"}\NormalTok{,}\StringTok{"can"}\NormalTok{)}
\end{Highlighting}
\end{Shaded}

Luego se representa en un grafico de barras, las 10 palabras mas
repetidas en los comentarios segun el sentimiento asignado al Tweet en
el que se encuentran.

\begin{Shaded}
\begin{Highlighting}[]
\NormalTok{tweets_expand <-}\StringTok{ }\NormalTok{tweets_expand }\OperatorTok\StringTok{ }\KeywordTok{filter}\NormalTok{(}\OperatorTok{!}\NormalTok{(word }\OperatorTok\StringTok{ }\NormalTok{lista_stopwords)) }

\NormalTok{tweets_expand }\OperatorTok\StringTok{ }\KeywordTok{group_by}\NormalTok{(Sentiment, word) }\OperatorTok\StringTok{ }
\StringTok{  }\KeywordTok{count}\NormalTok{(word) }\OperatorTok\StringTok{ }
\StringTok{  }\KeywordTok{group_by}\NormalTok{(Sentiment) }\OperatorTok\StringTok{ }
\StringTok{  }\KeywordTok{top_n}\NormalTok{(}\DecValTok{10}\NormalTok{,n) }\OperatorTok\StringTok{ }
\StringTok{  }\KeywordTok{arrange}\NormalTok{(Sentiment, }\KeywordTok{desc}\NormalTok{(n)) }\OperatorTok\StringTok{ }
\StringTok{  }\KeywordTok{ggplot}\NormalTok{(}\KeywordTok{aes}\NormalTok{(}\DataTypeTok{x=}\KeywordTok{reorder}\NormalTok{(word,n),}\DataTypeTok{y=}\NormalTok{n,}\DataTypeTok{fill=}\NormalTok{Sentiment)) }\OperatorTok{+}\StringTok{ }
\StringTok{  }\KeywordTok{geom_col}\NormalTok{() }\OperatorTok{+}\StringTok{ }
\StringTok{  }\KeywordTok{labs}\NormalTok{(}\DataTypeTok{y =} \StringTok{"Frecuencia"}\NormalTok{, }\DataTypeTok{x =} \StringTok{"Palabras mas repetidas"}\NormalTok{) }\OperatorTok{+}\StringTok{ }
\StringTok{  }\KeywordTok{coord_flip}\NormalTok{()}
\end{Highlighting}
\end{Shaded}

\includegraphics{hito_1_files/figure-latex/unnamed-chunk-19-1.pdf}

Se puede observar que, como era de esperarse, las palabras mas
comentadas en todas las categorias de sentimiento son las referentes
directamente a la pandemia: ``covid'' y ``coronavirus''. Se destaca de
igual forma la alta popularidad de las palabras ``food'' y ``prices'',
posiblemente debido al parcial desabastecimiento de productos y la
subida de precios producto de la cuarentena.

Un analisis similar se presenta a continuacion, pero esta vez en word
clouds:

\begin{Shaded}
\begin{Highlighting}[]
\CommentTok{#Listas de colores utilizadas en las nubes de palabras}
\NormalTok{pal_neg <-}\StringTok{ }\KeywordTok{c}\NormalTok{(}\StringTok{"#FC9272"}\NormalTok{, }\StringTok{"#FB6A4A"}\NormalTok{, }\StringTok{"#EF3B2C"}\NormalTok{, }\StringTok{"#CB181D"}\NormalTok{, }\StringTok{"#A50F15"}\NormalTok{)}
\NormalTok{pal_neu <-}\StringTok{ }\KeywordTok{c}\NormalTok{(}\StringTok{"#A1D99B"}\NormalTok{, }\StringTok{"#74C476"}\NormalTok{, }\StringTok{"#41AB5D"}\NormalTok{, }\StringTok{"#238B45"}\NormalTok{, }\StringTok{"#006D2C"}\NormalTok{)}
\NormalTok{pal_pos <-}\StringTok{ }\KeywordTok{c}\NormalTok{(}\StringTok{"#9ECAE1"}\NormalTok{, }\StringTok{"#6BAED6"}\NormalTok{, }\StringTok{"#4292C6"}\NormalTok{, }\StringTok{"#2171B5"}\NormalTok{, }\StringTok{"#08519C"}\NormalTok{)}

\NormalTok{most_frec <-}\StringTok{ }\KeywordTok{c}\NormalTok{(}\StringTok{"covid"}\NormalTok{,}\StringTok{"coronavirus"}\NormalTok{,}\StringTok{"supermarket"}\NormalTok{,}\StringTok{"grocery"}\NormalTok{,}\StringTok{"store"}\NormalTok{)}

\NormalTok{top <-}\StringTok{ }\DecValTok{400}
\end{Highlighting}
\end{Shaded}

\begin{Shaded}
\begin{Highlighting}[]
\CommentTok{#Palabras mas repetidas en comentarios negativos y extremadamente negativos}
\NormalTok{neg_tweets <-}\StringTok{ }\NormalTok{tweets_expand }\OperatorTok\StringTok{ }
\StringTok{  }\KeywordTok{filter}\NormalTok{(Sentiment }\OperatorTok{==}\StringTok{ "Negative"} \OperatorTok{|}\StringTok{ }\NormalTok{Sentiment }\OperatorTok{==}\StringTok{ "Extremely Negative"}\NormalTok{) }\OperatorTok
\StringTok{  }\KeywordTok{count}\NormalTok{(word)}
\NormalTok{neg_tweets_top <-}\StringTok{ }\NormalTok{neg_tweets}\OperatorTok\StringTok{ }
\StringTok{  }\KeywordTok{arrange}\NormalTok{(}\KeywordTok{desc}\NormalTok{(n)) }\OperatorTok
\StringTok{  }\KeywordTok{top_n}\NormalTok{(top,n) }


\KeywordTok{set.seed}\NormalTok{(}\DecValTok{1234}\NormalTok{)}
\KeywordTok{wordcloud}\NormalTok{(}\DataTypeTok{words =}\NormalTok{ neg_tweets_top}\OperatorTok{$}\NormalTok{word, }\DataTypeTok{scale=}\KeywordTok{c}\NormalTok{(}\DecValTok{5}\NormalTok{,}\FloatTok{0.7}\NormalTok{), }\DataTypeTok{freq =}\NormalTok{ neg_tweets_top}\OperatorTok{$}\NormalTok{n, }\DataTypeTok{min.freq =} \DecValTok{1}\NormalTok{,}\DataTypeTok{max.words=}\DecValTok{100}\NormalTok{, }\DataTypeTok{random.order=}\OtherTok{FALSE}\NormalTok{, }\DataTypeTok{rot.per=}\FloatTok{0.35}\NormalTok{, }\DataTypeTok{colors=}\NormalTok{pal_neg)}
\end{Highlighting}
\end{Shaded}

\includegraphics{hito_1_files/figure-latex/unnamed-chunk-21-1.pdf}

\begin{Shaded}
\begin{Highlighting}[]
\CommentTok{#Palabras mas repetidas en comentarios neutrales}
\NormalTok{neu_tweets <-}\StringTok{ }\NormalTok{tweets_expand }\OperatorTok\StringTok{ }
\StringTok{  }\KeywordTok{filter}\NormalTok{(Sentiment }\OperatorTok{==}\StringTok{ "Neutral"}\NormalTok{) }\OperatorTok
\StringTok{  }\KeywordTok{count}\NormalTok{(word)}
\NormalTok{neu_tweets_top <-}\StringTok{ }\NormalTok{neu_tweets}\OperatorTok
\StringTok{  }\KeywordTok{arrange}\NormalTok{(}\KeywordTok{desc}\NormalTok{(n)) }\OperatorTok
\StringTok{  }\KeywordTok{top_n}\NormalTok{(top,n) }
  

\KeywordTok{set.seed}\NormalTok{(}\DecValTok{1234}\NormalTok{)}
\KeywordTok{wordcloud}\NormalTok{(}\DataTypeTok{words =}\NormalTok{ neu_tweets_top}\OperatorTok{$}\NormalTok{word, }\DataTypeTok{scale=}\KeywordTok{c}\NormalTok{(}\DecValTok{5}\NormalTok{,}\FloatTok{0.7}\NormalTok{), }\DataTypeTok{freq =}\NormalTok{ neu_tweets_top}\OperatorTok{$}\NormalTok{n, }\DataTypeTok{min.freq =} \DecValTok{1}\NormalTok{,}\DataTypeTok{max.words=}\DecValTok{100}\NormalTok{, }\DataTypeTok{random.order=}\OtherTok{FALSE}\NormalTok{, }\DataTypeTok{rot.per=}\FloatTok{0.35}\NormalTok{, }\DataTypeTok{colors=}\NormalTok{pal_neu, }\DataTypeTok{fixed.asp =} \OtherTok{TRUE}\NormalTok{)}
\end{Highlighting}
\end{Shaded}

\includegraphics{hito_1_files/figure-latex/unnamed-chunk-22-1.pdf}

\begin{Shaded}
\begin{Highlighting}[]
\CommentTok{#Palabra mas repetidas en comentarios positivos y extremadamente positivos}
\NormalTok{pos_tweets <-}\StringTok{ }\NormalTok{tweets_expand }\OperatorTok\StringTok{ }
\StringTok{  }\KeywordTok{filter}\NormalTok{(Sentiment }\OperatorTok{==}\StringTok{ "Positive"} \OperatorTok{|}\StringTok{ }\NormalTok{Sentiment }\OperatorTok{==}\StringTok{ "Extremely Positive"}\NormalTok{) }\OperatorTok
\StringTok{  }\KeywordTok{count}\NormalTok{(word)}
\NormalTok{pos_tweets_top <-}\StringTok{ }\NormalTok{pos_tweets}\OperatorTok\StringTok{ }
\StringTok{  }\KeywordTok{arrange}\NormalTok{(}\KeywordTok{desc}\NormalTok{(n)) }\OperatorTok
\StringTok{  }\KeywordTok{top_n}\NormalTok{(top,n) }


\KeywordTok{set.seed}\NormalTok{(}\DecValTok{1234}\NormalTok{)}
\KeywordTok{wordcloud}\NormalTok{(}\DataTypeTok{words =}\NormalTok{ pos_tweets_top}\OperatorTok{$}\NormalTok{word, }\DataTypeTok{scale=}\KeywordTok{c}\NormalTok{(}\DecValTok{5}\NormalTok{,}\FloatTok{0.7}\NormalTok{), }\DataTypeTok{freq =}\NormalTok{ pos_tweets_top}\OperatorTok{$}\NormalTok{n, }\DataTypeTok{min.freq =} \DecValTok{1}\NormalTok{,}\DataTypeTok{max.words=}\DecValTok{100}\NormalTok{, }\DataTypeTok{random.order=}\OtherTok{FALSE}\NormalTok{, }\DataTypeTok{rot.per=}\FloatTok{0.35}\NormalTok{, }\DataTypeTok{colors=}\NormalTok{pal_pos, }\DataTypeTok{fixed.asp =} \OtherTok{TRUE}\NormalTok{)}
\end{Highlighting}
\end{Shaded}

\includegraphics{hito_1_files/figure-latex/unnamed-chunk-23-1.pdf}

Se puede destacar que el tipo de palabras utilizadas en los 3 contextos
de sentimiento (negativo, neutral y positivo) no varia en gran manera,
repitiendose tipicamente las mismas dentro de las mas populares:
``covid'', ``coronavirus'', ``supermarket'', ``food'', ``prices'',
``store'' y ``grocery''. Dejando de lado la palabra ``covid'' las
palabras mas repetidas se relacionan con el abastecimiento de productos
basicos, lo que permite inferir de forma preliminar que este tema fue
relevante para los usuarios durante la pandemia (entre marzo y abril).

En base a los datos obtenidos categorizados por sentimiento, puede
analizarse su largo promedio, y asi ver si existe alguna relacion entre
esas variables. Para esto se hace uso de los boxplot, con el fin de
comparar promedios y distribuciones, y detectar algunos outliers segun
el sentimiento. Es importante destacar aqui la presencia de hashtags
(\#), que suponen una mayor extension al largo del tweet.

\begin{Shaded}
\begin{Highlighting}[]
\CommentTok{#Promedio en el largo de tweets por sentimiento.}

\KeywordTok{ggplot}\NormalTok{(tweets, }\KeywordTok{aes}\NormalTok{(}\DataTypeTok{x=}\NormalTok{Sentiment, }\DataTypeTok{y =}\KeywordTok{str_length}\NormalTok{(OriginalTweet) ),) }\OperatorTok{+}\StringTok{ }\KeywordTok{geom_boxplot}\NormalTok{() }\OperatorTok{+}\StringTok{ }\KeywordTok{labs}\NormalTok{(}\DataTypeTok{y =} \StringTok{"Largo"}\NormalTok{, }\DataTypeTok{title=} \StringTok{"Distribucion del largo de tweets por sentimiento"}\NormalTok{)}
\end{Highlighting}
\end{Shaded}

\includegraphics{hito_1_files/figure-latex/unnamed-chunk-24-1.pdf}

\begin{Shaded}
\begin{Highlighting}[]
\KeywordTok{ggsave}\NormalTok{(}\DataTypeTok{plot =} \KeywordTok{last_plot}\NormalTok{(), }\DataTypeTok{width =} \DecValTok{10}\NormalTok{, }\DataTypeTok{height =} \DecValTok{10}\NormalTok{, }\DataTypeTok{dpi =} \DecValTok{300}\NormalTok{, }\DataTypeTok{filename =} \StringTok{"boxplot_len_words.png"}\NormalTok{)}
\end{Highlighting}
\end{Shaded}

Se puede observar que el largo de las palabras no parece ser
determinante para el valor de sentimiento del comentario en general,
pues en promedio todas miden muy parecido: aproximadamente 8 caracteres.

\hypertarget{propuesta-de-hipuxf3tesis}{%
\section{Propuesta de hipótesis}\label{propuesta-de-hipuxf3tesis}}

A partir del análisis exploratorio se proponen las siguientes hipótesis
y preguntas que se podrían abordar con este set de datos:

\begin{enumerate}
\def\labelenumi{\arabic{enumi}.}
\tightlist
\item
  ¿El sentimiento general sobre el COVID-19 varía por la localidad
  registrada?
\item
  Dada las características del COVID-19 y sus consecuencias, más del
  50\% de los tweets están asociados a un sentimiento negativo o
  extremadamente negativo.
\item
  ¿Los sentimientos mayormente expresados en los Tweets tienen relacion
  con el contexto social del lugar desde donde son publicados?
\item
  ¿Se puede asociar un sentimiento a una palabra dependiendo de las
  otras palabras mencionadas en un tweet?
\item
  ¿Existe un alza o descenso de comentarios positivos al avanzar de los
  días?¿Si es así, a qué se debe?
\item
  ¿Podemos generar un clasificador que se pueda generalizar en base a
  los datos que tenemos?
\item
  El sentimiento de los comentarios se ha visto modificado con respecto
  a tiempos anterior al Covid?
\end{enumerate}

\hypertarget{proximos-pasos.}{%
\section{Proximos pasos.}\label{proximos-pasos.}}

Como se mencionó antes, los datos manejados no tienen un estándar en la
categorización por localidad, por lo que resulta muy importante tratar
con nuevos dataset para agrupar mejor esas localizaciones.

Es necesario también mejorar el filtro de palabras, esta vez
considerando el uso de hashtags, para realizar un análisis más detallado
acerca de la correlación entre la extensión de estas y el sentimiento
asociado al tweet.

Y para entrar a comparar el sentimiento promedio de tweets en distintas
fechas, se requerirá elaborar y extender un algoritmo de clasificación a
nuevos dataset.

\hypertarget{referencias}{%
\section{Referencias}\label{referencias}}

{[}1{]} Twitter suma 166 millones de usuarios durante el Coronavirus.
(2020, 3 junio). REBOLD, Data-Driven Marketing \& Communication.
\url{https://letsrebold.com/es/blog/twitter-suma-166-millones-de-usuarios-frente-al-coronavirus/\#:\%7E:text=Asimismo\%2C\%20Twitter\%20comunic\%C3\%B3\%20en\%20la,el\%20primer\%20trimestre\%20de\%202019}.

{[}2{]} Coronavirus tweets NLP - Text Classification. (2020, 8
septiembre). Kaggle.
\url{https://www.kaggle.com/datatattle/covid-19-nlp-text-classification?select=Corona_NLP_test.csv}

\hypertarget{contribuciones-del-equipo}{%
\section{Contribuciones del equipo}\label{contribuciones-del-equipo}}

\begin{enumerate}
\def\labelenumi{\arabic{enumi}.}
\tightlist
\item
  Nicolás Herrera:
\item
  Yesenia Marulanda:
\item
  Franco Migliorelli:
\item
  Samuel Sánchez:
\item
  Sebastián Urbina:
\end{enumerate}

\end{document}
